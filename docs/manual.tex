\documentclass{article}

\usepackage{amsmath}
\usepackage{appendix}
\usepackage{hyperref}
\usepackage{siunitx}

\usepackage[english]{babel}

\begin{document}

\title{ATSPackage user manual}
\author{Vanessa McHale}
\maketitle

\tableofcontents

\section{Introduction}

ATSpackage is a collection of build scripts written in Haskell. There are three
things it accomplishes:

\begin{enumerate}
  \item Distributed builds. ATSPackage allows users depend on libraries that are
    hosted elsewhere.
  \item Simplified builds. As ATSPackage contains scripts to download the
    compiler, builds are easier for potential contributors.
  \item Haskell integration. ATSPackage has first-class support for building ATS
    code that depends on Haskell libraries. 
\end{enumerate}

With that in mind, it is worthwhile to enumerate some things that it does
\textit{not} accomplish:

\begin{enumerate}
  \item Package management. ATSPackage does allow for \textit{reproducible}
    builds, but it does not resolve dependencies. Future support for package
    management is planned.
  \item Full flexibility of C. As ATSPackage is intended to simplify builds, it
    does not expose everything. This will likely not cause problems, provided
    that the libraries dependend on are written in C, ATS, or Haskell.
\end{enumerate}

\section{Builds}

ATSPackage supports three build types: binary, dynamic library, and static
library.

\end{document}
