\documentclass{article}

\usepackage{amsmath}
\usepackage{appendix}
\usepackage{hyperref}
\usepackage{siunitx}

\usepackage[english]{babel}

\begin{document}

\title{ATSPackage user manual}
\author{Vanessa McHale}
\maketitle

\tableofcontents

\section{Introduction}

ATSpackage is a build tool for ATS written in Haskell. There are three
things it accomplishes:

\begin{enumerate}
  \item Distributed builds. ATSPackage allows users depend on libraries that are
    hosted elsewhere.
  \item Simplified builds. As ATSPackage contains scripts to download the
    compiler, builds are easier for potential contributors.
  \item Haskell integration. ATSPackage has first-class support for building ATS
    code that depends on Haskell libraries. 
\end{enumerate}

With that in mind, it is worthwhile to enumerate some things that it does
\textit{not} accomplish:

\begin{enumerate}
  \item Full flexibility of C. As ATSPackage is intended to simplify builds, it
    does not expose everything. This will likely not cause problems, provided
    that the libraries dependend on are written in C, ATS, or Haskell.
\end{enumerate}

\section{Package Resolution}

As ATS is a statically typed language, some form of dependency resolution is
necessary if we'd like to be able to share data structures between packages.

\section{Builds}

ATSPackage supports three build types: binary, dynamic library, and static
library.

\subsection{Binary Builds}

ATSPackage allows 

\subsection{Binary Builds with Haskell Dependencies}

ATSPackage allows binary builds with Haskell dependencies by allowing a package
to depend on an object file generated by GHC. The object file can be generated
by cabal, allowing full use of the Haskell ecosystem.

ATSPackage can also generate data types for ATS based on Haskell types. You can
use this to eliminate some of the work involved in writing FFI bindings, and
particularly to avoid ATS' lack of generics.

\end{document}
